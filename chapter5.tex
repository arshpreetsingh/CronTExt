\starttext 

\section{Discusion}

A major advantage of Clicked Cued Points is its large password space over alphanumeric passwords. The large password space is significant 
because it reduces the ”guessability” of passwords. Similarly, Clicked Cued Points has an advantage in password space over Blonder-style graphical 
passwords and recognition-based graphical password. But human usability is also an essential consideration. Our human factors testing of the Clicked 
Cued Points system in comparison to alphanumeric passwords yielded somewhat mixed results. In the learning phase the alphanumeric group took fewer 
trials to achieve 10 correct password inputs than did the graphical group. This is also reflected in significantly longer total times to input the 
graphical passwords. Seventy percent of the alphanumeric participants input the password 10 times without any errors, and all alphanumeric participants 
were able to achieve the criterion with a maximum of two incorrect password inputs. The graphical group took more trials and had more variability. 
Forty percent of graphical participants achieved input of the password 10 times without any errors, and 70 percent achieved the criterion with a 
maximum of three incorrect password inputs. Surprisingly, the least successful twenty percent of the group made between 17 and 20 incorrect password 
inputs. However. It should be clearly noted that most graphical participants did not have serious problems in the learning phase. We do not consider it 
a negative outcome that the graphical group made more errors that the alphanumeric group in learning. The graphical participants were using a new 
password system, completely unknown to them, for their first time. Understanding the password system and how to use it effectively took them time. 
Indeed, several participants in the graphical password group had significant difficulties in meeting the learning criterion. These participants may 
have had various problems, e.g., hand-eye coordination, lack of attention to precision in clicking, or poor choice of their graphical password. 
Whatever the reason for the long trails of incorrect practices, it should be recognized that the alphanumeric group had no equivalent challenges, given 
their long familiarity and use of alphanumeric passwords. The most common problem in graphical password input was clicking outside the tolerance around 
the user’s click point. Participants were often close to, but outside, the tolerance. Although at the end of the creation phase they viewed their click 
points with the tolerance outlined on the image, participants still had difficulty being as precise as required. The participants input their points 
slowly, so we believe this was not the result of lack of care in input. Rather, it appears that in password creation participants focused on the 
general area around their click point rather than the precise point. Thus, we believe that this was a memory problem. Graphical password users had to 
understand how much precision was needed to be successful and then train themselves to remember the password accordingly. For example, they might have 
needed to remember a password point as “the seat of the chair” rather than simply “the chair.” In the retention phase, the correctness of password 
inputs differed by trial for both groups. Since the R1 trial took place in the same session as the creation and learning phases, there were few bad 
inputs. In the R2 trial participants had more difficulty recalling their passwords, regardless of which group they were in. In the final R3 trial there 
appears to have been some consolidation of the passwords in memory because the incorrect inputs were lower than in R2 (though not significantly) in 
spite of the long time lapse. The lack of significant differences between the alphanumeric and graphical modes on the correctness of password inputs 
and lack of interactions between mode and trial, indicate that the main factor in correctness was password memory for both groups. It is encouraging 
that the graphical group was as to do as well or better than the graphical group, in their first experience with remembering graphical passwords. The 
time for the correct input of the password showed that the alphanumeric group was faster in all three retention trials. We expected a longer input time 
for the graphical group based on the time for mouse movement and selection of the target . However, the large difference between the two groups points 
to a conclusion that the time difference was mostly a result of think time to locate the correct click region and determine precisely where to click in 
it. Among the three sessions, the fastest input was in R1, a result that was expected and reflected the high correctness of inputs in R1. It is 
interesting to note that in R1 the difference in input times of the alphanumeric and graphical groups were not very large, about 3.5 seconds longer in 
the graphical group. This outcome was achieved with graphical participants who were certainly not automated in the graphical input process. This 
suggests that with highly skilled users the input time will not be much longer than alphanumeric input times. The increased time for the correct 
password input in R2 and R3 was quite elevated over R1 for both alphanumeric and graphical groups. This shows the effect of intermittent use on memory 
for passwords. In addition, the graphical group in R2 was by far the slowest, indicating that participants proceeded slowly and carefully to ensure 
correct recall and input. Interestingly, in the graphical group, the R3 trial at the end of five weeks was a bit faster (though not significantly) than 
the R2 trial. This may mean that the graphical participants were becoming familiar enough with their graphical passwords and input procedures to work 
more quickly.

\section{Conclusion}


The empirical testing of Clicked Cued Points indicates strengths and weaknesses, but is overall encouraging. Graphical users’ retention of their 
password over five weeks was similar to alphanumeric users, perhaps even a bit better, This result is notable because it was achieved in very 
intermittent use and with very little experience with graphical passwords. In practice users of graphical passwords may exceed alphanumeric password 
users, given more experience with graphical passwords and the opportunity to use their graphical passwords regularly for some period of time. While 
graphical users always took more time to input their passwords than alphanumeric users, even so there was evidence that with continuous use graphical 
passwords can be entered quite quickly. This work focused on the usability of Clicked Cued Points, but its security is also an important issue. Clicked 
Cued Points seems to hold out the prospect of a much more secure system. It is easy to obtain large passwords spaces. Furthermore, in our experiment it 
appears that users rarely chose points that were within the tolerance around the click point of another participant. That is, people were not strongly 
drawn to a few salient areas that an attacker might guess. Finally, there is currently no efficient way of creating dictionary attacks against the 
system. These observations point to further study of the security and usability of Clicked Cued Points. From the viewpoint of security, we plan to 
study the potential for new kinds of dictionary attacks against graphical passwords. For example, edge detection techniques might be used to find out 
whether graphical passwords can be attacked by exploiting orderly ways in which users choose passwords on an image. From the viewpoint of usability, we 
are interested in determining the effect of the particular image used on success with graphical passwords, studying users’ speed in skilled 
performance, and discovering what kinds of insecure password practices users invent for graphical passwords.


\section{Future work}


This dissertation introduced the prediction of the behaviour of users that are surfing various sites on the web based on various parameters. There are 
number of areas into which proposed work can be extended. The work shows how to generate cluster automatically but there is a great impact of Distance 
Determination Factor (DDF) on the number of Algorithm to be formed. This work is sensitive to initial seed selection. This can be improved by carefully 
examining the relationships between data objects beforehand. Outlier detection and removal is another area where work can be done. There must be some 
method to detect the outliers and can be removed if desired. The research could be extended in this direction to revise the partitioning based 
clustering algorithm, which can reduce the complexity of the proposed algorithm.



\stoptext
