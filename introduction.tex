\starttext 
Human factors often considered weakest link in computer Security system. Computer security is necessary. Many Security researchers point out 
that there are three major areas where human computer interaction is important : Authentication, security operations and developing secure systems. 
There is currently the phenomenon of threats at the threshold of the internet, internal networks and secure environments. Although security researchers 
have made great strides in fighting these threats by protecting systems, individual users and digital assets, unfortunately the threats continue to 
cause problems. The principle area of attack is AUTHENTICATION, which is of course the process of determining the accessibility of a user to a 
particular resource or system.Here we focus on authentication problem. Authentication in the computer world refers to the act of confirming the 
authenticity of the user's digital identity claim. Currently, popular authentication mechanisms are mainly based the following factors: something that 
the user has (an object), knows (a secret), or uniquely represents him (biometric identifiers) [1]. In the simplest form, a system that requires 
authentication challenges the user for a secret, typically a pair of username and password. The entry of the correct pair grants access on the system’s 
services or resources. Unfortunately,
 this approach is susceptible to several vulnerabilities and drawbacks. These shortcomings range from user selected weak or easily guessable passwords 
to more sophisticated threats such as malware and keyboard sniffers [2]. An adversary has an abundance of opportunities to compromise the text-based 
password authentication mechanisms. For long time the computer industry has been in a quest for better alternatives but without popular success: most 
of our current systems still use the primitive text-based authentication schemes.
 For the vast majority of computer systems, pass words is the method of choice for authenticating users. The most widely and commonly used 
authentication is traditional “Username” and “Password”. Today, passive or active users are the key consideration of security mechanisms. Th e passive 
user is only interested in understanding the system . The active user, on the other hand, will consider and reflect o n ease of use, efficiency , 
memorability, effectiveness and satis faction of the system. For such , Aut hentication Methods can be divi ded into three main area s: Token Based 
authentication, Biomet ric based authentication , Knowledge based auth entication. Token based techniques, such as key cards , bank cards and smart 
cards are widely us ed. Many token-based authen tication systems also use knowledge
 based techniques to enhance securit y. For example, ATM cards are general ly used together with a PIN number. Biometric based authentication tec 
hniques, such as fingerprints, iris scan, or faci al recognition, are not yet widely ado pted. The major drawback of this approach is that such systems 
can be expe nsive, and the identification process can be slow and often unreliable. However, this type
 of technique provides the highest level of security.

\section{Problems in Text based password scheme}

Knowledge based techniques are the most widely used authentication techniques and include both text-based and picture-based passwords.Graphical Password can be formed in the combination of Image Icons or Pictures.  In other words, graphical password is an authentication system that 
works by having the user select from images, in a specific order, presented in a graphical user interface (GUI).  For this reason, the 
graphical-password approach is sometimes called graphical user authentication (GUA).  Computer and Information security is very much dependent on 
password for the authentication of the users and are common in practice. The password design methods include text method, Biometrics. Biometrics scheme 
cannot be used widely. Text method is most widely used, since it is easy to implement and use. One of the main limitation in text-based password is the 
difficulty of remembering it. Studies have shown that users tend to pick short and easy passwords that can be used by them easily.  But, these 
passwords can also be easily guessed or broken. Text based password scheme is lacking the above essential points mostly. Graphical based passwords 
might be a solution to the problems. Greg Blonder pioneered the idea of graphical passwords in 1996. His idea is to let the user click (with a mouse or 
stylus) on a few chosen (predesigned) regions in (pre-processed) an image that appears on the screen. Problems in Text based password scheme User may 
forget the password if it is too long or complicated or the password remain unused for a long time. Watching a user log on as they type their password.  
Dictionary attacks.

\section{Graphical password scheme}


It is a better alternative to text based password scheme. To amend the some of the shortcomings of the textual passwords, researchers turned their 
attention to passwords that utilize graphical objects [3, 4, 5]. Graphical authentication has been proposed as a user-friendly alternative to password 
generation and authentication [6, 7]. The main difference to textual passwords is the use if a device with graphical input: the user enters the 
password by clicking on a set of images, specific pixels of an image, or by drawing a pattern in a pre-defined and secret order. The proposed systems 
claim to provide a superior space of possible password combinations compared to traditional 8-character textual passwords [4]. This property alone 
renders attacks including dictionary attacks and keyboard sniffers computationally hard increasing our ability to defend against brute-force attacks. 
Furthermore, according to Picture Superiority Effect Theory [8], concepts are more likely to be recognized and remembered if they are presented as 
pictures rather than as words. Thus, graphical password presumably delivers a higher usability compare to text-based password. Pictures are generally 
easier to be remembered than text. If the number of pictures is sufficiently large, the possible password space of a graphical password scheme may 
exceed that of text based schemes and hence may prove to offer better resistance to dictionary attacks. Examples include places we visited, faces of 
people and things we have seen which are easy to reframe. Difficult to implement automated attacks (such as dictionary attacks) against graphical 
passwords. The picture-based techniques can be further divided into two categories: recognition-based and recall-based graphical techniques. Using 
recognition-based techniques, a user is presented with a set of images and the user passes the authentication by recognizing and identifying the images 
he or she selected during the registration stage. Using recall-based techniques, a user is asked to reproduce something that he or she created or 
selected earlier during the registration stage.
The past decade has seen a growing interest in using graphical passwords as an alternative tothe traditional text-based passwords. Researchers are 
continually introduci ng new ideas,concepts, and features in the field of graphical authentication. There are already manyapproaches have been proposed 
in present times. Graphical password techniques can beclassified into two categories: recognition-based and recall-based. In recognition-basedsystems, 
a series of images are presented to the user and a successful authenticationrequires a correct images being clicked in a right order. In recall-based 
systems, the useris asked to reproduce something that he or she created or selected earlier during theregistration. Blonder gave the initial idea of 
graphical password in 1996. In his scheme, a user ispresented with one predetermined image on a visual display and required to select one or 
morepredetermined positions on the displayed image in a particular order to access the restrictedresource [5]. The major drawback of this scheme is 
that users cannot click arbitrarily on thebackground. The memorable password space was not studied by the author either. Wiedenbeck et al [6] proposed 
PassPoint method in which they extended Blonder’s idea by eliminating the
International Journal of Computer Engineering and Technology (IJCET), ISSN 0976 – 6367(Print),ISSN 0976 – 6375(Online)
Volume 3, Issue 2, July- Septembe r (2012), © IAEME 
predefined boundaries and allowing arbitrary images to be used. As a result, a user can click onany place on an image (as opposed to some pre-defined 
areas) to create a password. A tolerancearound each chosen pixel is calculated. In order to be authenticated, the user must click withinthe tolerance 
of their chosen pixels and also in the correct sequence. Few grid based schemes areproposed which uses recall method. Jermyn et al [7] proposed a 
technique called “Draw ASecret” (DAS) where a user draws the password on a 2D grid. The coordinates of thisdrawing on the grid are stored in order. 
During authentication user must redraw the picture.The user is authenticated if the drawing touches the grid in the same order. The major drawback of 
DAS is that diagonal lines are difficult to draw and difficulties might arise when theuser chooses a drawing that contains strokes that pass too close 
to a grid-line. Users haveto draw their input sufficiently away from the grid lines and intersections in order to enter thepassword correctly. If a 
user draws a password close to the grid lines or intersections, thescheme may not distinguish which cell the user is choosing. Syukri et al. [8] 
proposed a systemwhere authentication is conducted by having the user drawing his/her signature using a mouse.Dhamija and Perrig [9] proposed a 
graphical authentication scheme in which the user selectscertain number of images from a set of random pictures during registration. Later user has 
toidentify the pre-selected images for authentication. The users are presented a set of pictures onthe interface, some of them taken from their 
portfolio, and some images selectedrandomly. For successful authentication, users have to select ‘their’ pictures amongst thedistracters. Passface is a 
technique developed by Real User Corporation [10]. The basic idea issame as Dhamija and Perrig method. Here the user is asked to choose four images of 
humanfaces from a face database as their password. There is a common weakness in the abovegraphical password schemes: they are all vulnerable to 
shoulder-surfing attacks. To address thisissue, Sobrado and Birget developed a graphical password technique [11]. In their scheme, thesystem first 
displays a number of 3 pass-objects (pre-selected by a user) among many otherobjects. To be authenticated, a user needs to recognize pass-objects and 
click inside the triangleformed by the 3 pass-objects. Man, et al. [12] proposed another shoulder-surfing resistantalgorithm in which a user selects a 
number of pictures as pass-objects. Each pass-object hasseveral variants and each variant is assigned a unique code. During authentication, the user 
ischallenged with several scenes. Each scene contains several pass-objects and many decoy-objects. The user has to type in a string with the unique 
codes corresponding to the pass-object.However, these methods force the user to memorize too many text strings, and their shoulder-surfing resistant 
property is not strong either. Chiasson et al [13] proposed a cued-recallgraphical password technique. Users click on one point per image for a 
sequence of images. Thenext image displayed is based on the previous click point so users receive implicit feedback as to whether they are on the 
correct path when logging in. A wrong click leads down anincorrect path, with an explicit indication of authentication failure only after the final 
click.The visual cue does not explicitly reveal right or wrong but is evident using knowledge only thelegitimate user should possess. These techniques 
have the potential to fill the gaps left betweentraditional authentication techniques, including trade-offs between security levels, expense anderror 
tolerance [14]. But unfortunately in real scenario, these approaches are under-utilized as theauthentications are usually complex and boring for users.




\subsection{Usability Features} 

In human computer interaction and computer science, usability usually refers to the eleganceand clarity with which the interaction 
with a computer program or a web site is designed. Typicaldictionary definitions show ‘usability’ to be a noun to the adjective ‘usable’, which means 
thatsomething is capable of being used, or is convenient and practicable for use.
  
\subsection{Password Strength} 

Wikipedia define the password strength as the likelihood that a password can be guessed by anunauthorized person or computer. 
Passwords easily guessed are known as weak or vulnerable;passwords very difficult or impossible to guess are considered strong. The terms weak and 
strongare relative and have meaning only with regard to specific password systems. T he necessaryquality of the password depends on how well the 
password system limits attempts to guess auser's password, whether by a person who knows the user well, or a computer trying millions of possibilities. 
In this section, we will discuss the password strength of our scheme byconsidering both the password space of the text password and the graphical 
password parts.The p assword space of first round authentication will be R
where p is themaximum number of images used to form the password.Now consider the round-2 authentication. Let M×N be the size of the image portfolio 
andsuppose the size of the allowable click-area is n×n for any POI. So the available numbers of POIare ×௡ మ . A text password of maximum length q
International Journal of Computer Engineering and Technology (IJCET), ISSN 0976 – 6367(Print),ISSN 0976 – 6375(Online) Volume 3, Issue 2, July- 
September (2012), © IAEME 585
characters are selected uniformly at random and independently from an alphabet of C characters.Naturally the password space of second round 
authentication.
 
\subsection{Resisting Brute Force and Dictionary Attacks} 

Dictionary attack is normally used to crack the text password. In hybrid scheme, first 
roundauthentication is based on graphical password and second round also involves mouse input, so itwill be impractical to carry out dictionary attacks 
against this scheme. Effort required in brute-force is directly proportional to the password space. In our scheme password space is very large.So an 
enormous effort is required to apply a brute force attack against our scheme. The attack programs need to automatically generate accurate 
click-positions to imitate human input, whichis particularly difficult for this scheme.
 
\subsection{Mitigating Keylogger Attacks}

Keylogging is a common method for 
stealing user text passwords. A keylogger is malicioussoftware which 
intercepts keystrokes on an infected machine as a user types. For 
example,Microsoft Windows provides (un-documented) interfaces 
facilitating interception of systemevents including keystrokes. With our 
hybrid scheme, a user would use the keyboard for the textpassword part, 
and mouse clicks for the graphical parts. Thus, a naive keylogger cannot 
obtainthe graphical parts. More sophisticated malware can capture both 
user screen contents and mouseclicks to recover a graphical password, 
with more effort. But it is not clear whether “mousetracking” spyware 
will be an effective tool against graphical passwords. However, mouse 
motionalone is not enough to break graphical passwords. Such information 
has to be correlated withapplication information, such as window 
position and size, as well as timing information.
 
\subsection{Mitigating Phishing Attacks}


Phishing is another common technique for 
stealing passwords by fooling users to enter suchinformation into a 
fraudulent website spoofing a legitimate one (e.g., a bank site). 
Socialengineering tactics are often used (e.g., “urgent account update”, 
requests to verify faketransactions, etc.). In hybrid system, the image 
portfolio of first round authentication may besame to all users. So an 
expert phisher can steal the password of first part easily. But in 
secondround authentication, it is very difficult to crack even users’ 
text password. Our text password isassociated with a POI which is 
graphical part. Without the knowledge of users’ image profile, thephisher 
does not know what images to present in order to extract a POI. So 
obtaining its round-2password is very difficult.Resisting Shoulder Surfing Attacks: Like text based passwords, most of 
the graphical passwords are vulnerable to shoulder surfing.At this 
point, our policy provides a solid resistance against shoulder surfing 
attack. In first roundauthentication, the user is presented with a set 
of 25 images. Naturally the size of each imagewill be very small and 
each login time the order of the images will be random. This technique 
isspecially deployed to make an imposter confused who is trying to 
memorize the authenticationdetails from the backside .




\section{Using CAPTCHA in Graphical Password Scheme}

Text-based password schemes have inherent security and usability 
problems, leading to the development of graphical password schemes. 
However, most of these alternate schemes are vulnerable to spyware 
attacks. We propose a new scheme, using CAPTCHA (Completely Automated 
Public Turing tests to tell Computers and Humans Apart) that retaining 
the advantages of graphical password schemes, while simultaneously 
raising the cost of adversaries by orders of magnitude. Furthermore, 
some primary experiments are conducted and the results indicate that the 
usability should be improved in the future work.


\section{NUMERIC AND GRAPHICAL SCHEME}


Day by day number of Internet users increasing. Now people are using different online services provided by Banks, College/Schools, Hospitals, online 
utility bill payment and online shopping sites. To access online services Text-based authentication system is in use. The text-based authentication 
scheme faces some drawbacks with usability and security issues that bring troubles to users. For example, if the user is not very intelligently 
constructed the password with extra security measures, it is very easy to hack for an expert hacker. On the contrary, if a password is hard to guess, 
then it is often hard to remember. A person has to memorize as many password as many different websites he/she is using. So he/she gets confused and/or 
forgets the correct userId/password combinations. We should have an alternative system to overcome these problems. To deal with these drawbacks, 
authentication scheme that use a combination of images as password is proposed. Graphical passwords consist of clicking or dragging activities on the 
pictures rather than typing textual characters, might be the option to overcome the problems that arises from the Text-based password system. In this 
paper, a comprehensive study of the existing graphical password schemes and shoulder surfing problem is performed. The best way in asynchronous mode 
user/password validation and One Time Password authentication is proposed for enhancement in security and privacy.

\section{Click-Draw Based Graphical Password Scheme}


Nowadays, graphical password is being regarded as a promising alternative in network security to replace traditional text-based password in which users 
interact with images for authentication rather than input alphanumeric stings. In general, this image-based authentication can be classified into three 
categories: click-based graphical password, choice-based graphical password and draw-based graphical password. However, each of them suffers from 
several intrinsic limitations. In this paper, we propose and develop a click-draw based graphical password scheme (CD-GPS) with the purpose of 
improving the image-based authentication in both security and usability by combining the above three techniques. Specifically, our scheme mainly 
contains two operational steps: image selection and secret drawing. That is, users first choose an ordered sequence of images and then select some of 
them to click-draw their secrets. We present an initial user study which shows positive results that our scheme is good at both security and usability, 
and subsequently give a preliminary security analysis of our scheme against several well-known attacks (e.g., dictionary attack).

\section{Hybrid Password Scheme}


INTERNATIONAL JOURNAL OF COMPUTER ENGINEERING & International Journal of Computer Engineering and Technology (IJCET)  Asraful Haque, Babbar Imam, Nes
ar Ahmad 
Department of Computer Engineering, Aligarh Muslim 
University, U.P.-202002, India asrafb4u@gmail.com, babbarimam5@gmail.com, nesar.ahmad@gmail.com ABSTRACT The most common computer authentication method 
is to use alphanumerical usernames and passwords. This method has been shown to have significant drawbacks. They have problems such as being hard to 
remember, vulnerable to guessing, phishing, dictionary attack, key-logger, and social engineering. Researchers have come out with an alternate password 
scheme called graphical password where they tried to improve the security and avoid the weakness of conventional password. Psychological studies say 
that human can remember pictures better than text. But graphical password scheme also has several drawbacks like shoulder-surfing problem, more storage 
space required and hard to implement compared to text passwords. In this paper, we have suggested a hybrid authentication system combining graphical 
and text passwords. User authentication has been verified in two steps to increase the security. We believe that in our system, users will be able to 
select stronger passwords through better user interface design. Keywords: Graphical Passwords, Text Passwords, User Authentication, Security, Password- 
Space I. INTRODUCTION AUTHENTICATION refers to the process of confirming or denying an individual’s claimed identity. Passwords are the most common 
means of authentication which do not require any special hardware. Typically passwords are strings of letters and digits. Alphanumeric passwords are 
versatile and easy to implement and use. They are required to satisfy two contradictory requirements. They have to be easily remembered by a user, 
while they have to be hard to guess by impostor [1]. Users are known to choose easily guessable and/or short text passwords, which are an easy target 
of dictionary and brute-forced attacks [2, 3]. Enforcing a strong password policy sometimes leads to an opposite effect, as a user may resort to write 
his or her difficult-to-remember passwords on sticky notes exposing them to direct theft [4]. Text 579 International Journal of Computer Engineering 
and Technology (IJCET), ISSN 0976 – 6367(Print),ISSN 0976 – 6375(Online) Volume 3, Issue 2, July- September (2012), © IAEMEpasswords can be stolen by 
malicious software (e.g., keystroke loggers) when being entered fromkeyboards. Phishing is another serious threat to text passwords, by which, a user 
could bepersuaded to visit a forged website and enter their passwords. To overcome the problemsassociated with text-passwords, the researchers have 
proposed the concept of graphicalpasswords. Graphical passwords schemes are the most promising alternative to conventionalpassword authentication 
systems. Graphical passwords employ graphical presentations such asicons, human faces or custom images to create a password. If the number of possible 
picturesis sufficiently large, the possible password space of a graphical password scheme mayexceed that of text-based password and therefore it is 
virtually more resistance to attackssuch as dictionary attacks. An important advantage of graphical passwords is that they are easierto remember than 
textual passwords due to the fact that human brains can process graphicalimages easily. Human brains have the ability to remember faces of people, 
places they visitand things they have seen for a longer duration. In this way graphical passwords provide ameans for making more user-friendly 
passwords while increasing the level of security. Besidesthese advantages, the most common problem with graphical passwords is the shoulder 
surfingproblem: an onlooker can steal user’s graphical password by watching in the user’s vicinity.Many researchers have attempted to solve this 
problem by providing different techniques.Another common problem with graphical passwords is that it takes longer to input graphicalpasswords than 
textual passwords. The login process is slow and it may frustrate theimpatient users. Since the graphical password is not widely deployed in real 
systemsvulnerabilities of graphical passwords are still not fully understood. In modern era ofcomputerization, a big necessity to have a strong 
authentication method is needed to secure allour application as much as possible. In this paper, considering the problems of text-password and 
graphical passwords, we haveproposed a novel hybrid password scheme which has desirable usability and security features.The structure of our paper is 
organized as follows. Section 2 reviews few existing graphicalauthentication methods. Section 3 explains our new proposed scheme. Section 4 analyzes 
thescheme in different dimensions. Section 5 concludes the paper with some remarks.II. GRAPHICAL PASSWORD SCHEMES The past decade has seen a growing 
interest in using graphical passwords as an alternative tothe traditional text-based passwords. Researchers are continually introducing new 
ideas,concepts, and features in the field of graphical authentication. There are already manyapproaches have been proposed in present times. Graphical 
password techniques can beclassified into two categories: recognition-based and recall-based. In recognition-basedsystems, a series of images are 
presented to the user and a successful authenticationrequires a correct images being clicked in a right order. In recall-based systems, the useris 
asked to reproduce something that he or she created or selected earlier during theregistration. Blonder gave the initial idea of graphical password in 
1996. In his scheme, a user ispresented with one predetermined image on a visual display and required to select one or morepredetermined positions on 
the displayed image in a particular order to access the restrictedresource [5]. The major drawback of this scheme is that users cannot click 
arbitrarily on thebackground. The memorable password space was not studied by the author either. Wiedenbecket al [6] proposed PassPoint method in which 
they extended Blonder’s idea by eliminating the 580 International Journal of Computer Engineering and Technology (IJCET), ISSN 0976 – 6367(Print),ISSN 
0976 – 6375(Online) Volume 3, Issue 2, July- September (2012), © IAEMEpredefined boundaries and allowing arbitrary images to be used. As a result, a 
user can click onany place on an image (as opposed to some pre-defined areas) to create a password. A tolerancearound each chosen pixel is calculated. 
In order to be authenticated, the user must click withinthe tolerance of their chosen pixels and also in the correct sequence. Few grid based schemes 
areproposed which uses recall method. Jermyn et al [7] proposed a technique called “Draw ASecret” (DAS) where a user draws the password on a 2D grid. 
The coordinates of thisdrawing on the grid are stored in order. During authentication user must redraw the picture.The user is authenticated if the 
drawing touches the grid in the same order. The major drawbackof DAS is that diagonal lines are difficult to draw and difficulties might arise when 
theuser chooses a drawing that contains strokes that pass too close to a grid-line. Users haveto draw their input sufficiently away from the grid lines 
and intersections in order to enter thepassword correctly. If a user draws a password close to the grid lines or intersections, thescheme may not 
distinguish which cell the user is choosing. Syukri et al. [8] proposed a systemwhere authentication is conducted by having the user drawing his/her 
signature using a mouse.Dhamija and Perrig [9] proposed a graphical authentication scheme in which the user selectscertain number of images from a set 
of random pictures during registration. Later user has toidentify the pre-selected images for authentication. The users are presented a set of pictures 
onthe interface, some of them taken from their portfolio, and some images selectedrandomly. For successful authentication, users have to select ‘their’ 
pictures amongst thedistracters. Passface is a technique developed by Real User Corporation [10]. The basic idea issame as Dhamija and Perrig method. 
Here the user is asked to choose four images of humanfaces from a face database as their password. There is a common weakness in the abovegraphical 
password schemes: they are all vulnerable to shoulder-surfing attacks. To address thisissue, Sobrado and Birget developed a graphical password technique 
[11]. In their scheme, thesystem first displays a number of 3 pass-objects (pre-selected by a user) among many otherobjects. To be authenticated, a user 
needs to recognize pass-objects and click inside the triangleformed by the 3 pass-objects. Man, et al. [12] proposed another shoulder-surfing 
resistantalgorithm in which a user selects a number of pictures as pass-objects. Each pass-object hasseveral variants and each variant is assigned a 
unique code. During authentication, the user ischallenged with several scenes. Each scene contains several pass-objects and many decoy-objects. The 
user has to type in a string with the unique codes corresponding to the pass-object.However, these methods force the user to memorize too many text 
strings, and their shoulder-surfing resistant property is not strong either. Chiasson et al [13] proposed a cued-recallgraphical password technique. 
Users click on one point per image for a sequence of images. Thenext image displayed is based on the previous click point so users receive implicit 
feedbackas to whether they are on the correct path when logging in. A wrong click leads down anincorrect path, with an explicit indication of 
authentication failure only after the final click.The visual cue does not explicitly reveal right or wrong but is evident using knowledge only 
thelegitimate user should possess. These techniques have the potential to fill the gaps left betweentraditional authentication techniques, including 
trade-offs between security levels, expense anderror tolerance [14]. But unfortunately in real scenario, these approaches are under-utilized as 
theauthentications are usually complex and boring for users. 581 International Journal of Computer Engineering and Technology (IJCET), ISSN 0976 – 
6367(Print),ISSN 0976 – 6375(Online) Volume 3, Issue 2, July- September (2012), © IAEMEIII. OUR PROPOSAL Given that text passwords are easy to deploy 
and to use, we believe that they will continue tobe popular. Graphical passwords are new and have some advantages over text password. Thus,we suggest 
that a combinational scheme of both text passwords and graphical passwords shouldbe made to enhance the security by addressing common password attacks. 
To this end, wepropose Two-Round Hybrid scheme. The proposed authentication system is divided into twophases as follows.A. Registration Phase:1. A user 
creates his profile by entering personal details and username.2. Then he is presented with a set of 25 images as shown in Fig 1. This is the common 
image- set for all users. Figure 1: Image-set for registration The user has to select any number of images from this set. Even he may choose a single 
image more than once. This selection will act as the password of his first round of authentication.3. Next he will choose any picture from the stored 
image database or from the local memory at his own choice.4. Now the user will select a point in the image and then type a text password. Password will 
be associated with that Point of Interest (POI). Each POI is described by a square (center and some tolerance in both X and Y axis). 582 International 
Journal of Computer Engineering and Technology (IJCET), ISSN 0976 – 6367(Print),ISSN 0976 – 6375(Online) Volume 3, Issue 2, July- September (2012), © 
IAEMEB. Login Phase:1. In round-1, a user is asked for his user name and graphical password (correct selection of images in a correct sequence). The 
order of images within the set will be random at every login time. This authentication step is shown in Fig 2.2. After supplying this, and independent 
of whether or not it is correct, in round two authentications, the user is presented with the pre-selected image.3. Here the user first clicks on the 
pre-defined POI. It is not possible for anyone to choose the exact point. We have to assign an acceptable tolerance to POI to minimize the false 
positives and false negatives. For an actual point (x, y) we allow the user to click any point which has the X-coordinate in between (x-5) to (x+5) and 
Y-coordinate in between (y-5) to (y+5). It means the allowable click-area will be a square of length10.4. After selection of correct POI user is 
presented with a text box where he has to type the text password as shown in Fig 3.5. After the successful entries in both rounds the user is allowed 
to access his account. Figure 2: Round-1 Authentication 583 International Journal of Computer Engineering and Technology (IJCET), ISSN 0976 – 
6367(Print),ISSN 0976 – 6375(Online) Volume 3, Issue 2, July- September (2012), © IAEME Figure 3: Round-2 AuthenticationIV. ANALYSIS OF THE SCHEME A. 
Usability Features: In human computer interaction and computer science, usability usually refers to the eleganceand clarity with which the interaction 
with a computer program or a web site is designed. Typicaldictionary definitions show ‘usability’ to be a noun to the adjective ‘usable’, which means 
thatsomething is capable of being used, or is convenient and practicable for use. Our scheme has thefollowing usability features: • Easy to use and 
memorize • Easy to create the password • Design and view mode is acceptable B. Password Strength: Wikipedia define the password strength as the 
likelihood that a password can be guessed by anunauthorized person or computer. Passwords easily guessed are known as weak or vulnerable;passwords very 
difficult or impossible to guess are considered strong. The terms weak and strongare relative and have meaning only with regard to specific password 
systems. The necessaryquality of the password depends on how well the password system limits attempts to guess ausers password, whether by a person who 
knows the user well, or a computer trying millions ofpossibilities [15]. In this section, we will discuss the password strength of our scheme 
byconsidering both the password space of the text password and the graphical password parts. ௣ The password space of first round authentication will be 
R1= ෌௜ୀଵ 25௜ ; where p is themaximum number of images used to form the password. Now consider the round-2 authentication. Let M×N be the size of the 
image portfolio andsuppose the size of the allowable click-area is n×n for any POI. So the available numbers of POI ெ×ே ௤are ௡మ . A text password of 
maximum length q characters has password space෌௝ୀଵ ‫ܥ‬௝ , if 584 International Journal of Computer Engineering and Technology (IJCET), ISSN 0976 – 
6367(Print),ISSN 0976 – 6375(Online) Volume 3, Issue 2, July- September (2012), © IAEMEcharacters are selected uniformly at random and independently 
from an alphabet of C characters. ெ×ே ௤Naturally the password space of second round authentication will be R2 = మ ×෌௝ୀଵ ‫ܥ‬௝ . ௡ If we combine these two 
steps, we will achieve the overall password space of our schemewhich is P = R1 × R2 . C. Resisting Brute Force and Dictionary Attacks: Dictionary 
attack is normally used to crack the text password. In hybrid scheme, first roundauthentication is based on graphical password and second round also 
involves mouse input, so itwill be impractical to carry out dictionary attacks against this scheme. Effort required in brute-force is directly 
proportional to the password space. In our scheme password space is very large.So an enormous effort is required to apply a brute force attack against 
our scheme. The attackprograms need to automatically generate accurate click-positions to imitate human input, whichis particularly difficult for this 
scheme. D. Mitigating Keylogger Attacks: Keylogging is a common method for stealing user text passwords. A keylogger is malicioussoftware which 
intercepts keystrokes on an infected machine as a user types. For example,Microsoft Windows provides (un-documented) interfaces facilitating 
interception of systemevents including keystrokes. With our hybrid scheme, a user would use the keyboard for the textpassword part, and mouse clicks 
for the graphical parts. Thus, a naive keylogger cannot obtainthe graphical parts. More sophisticated malware can capture both user screen contents and 
mouseclicks to recover a graphical password, with more effort. But it is not clear whether “mousetracking” spyware will be an effective tool against 
graphical passwords. However, mouse motionalone is not enough to break graphical passwords. Such information has to be correlated withapplication 
information, such as window position and size, as well as timing information. E. Mitigating Phishing Attacks: Phishing is another common technique for 
stealing passwords by fooling users to enter suchinformation into a fraudulent website spoofing a legitimate one (e.g., a bank site). Socialengineering 
tactics are often used (e.g., “urgent account update”, requests to verify faketransactions, etc.). In hybrid system, the image portfolio of first round 
authentication may besame to all users. So an expert phisher can steal the password of first part easily. But in secondround authentication, it is very 
difficult to crack even users’ text password. Our text password isassociated with a POI which is graphical part. Without the knowledge of users’ image 
profile, thephisher does not know what images to present in order to extract a POI. So obtaining its round-2password is very difficult. F. Resisting 
Shoulder Surfing Attacks: Like text based passwords, most of the graphical passwords are vulnerable to shoulder surfing.At this point, our policy 
provides a solid resistance against shoulder surfing attack. In first roundauthentication, the user is presented with a set of 25 images. Naturally the 
size of each imagewill be very small and each login time the order of the images will be random. This technique isspecially deployed to make an 
imposter confused who is trying to memorize the authenticationdetails from the backside. 585 International Journal of Computer Engineering and 
Technology (IJCET), ISSN 0976 – 6367(Print),ISSN 0976 – 6375(Online) Volume 3, Issue 2, July- September (2012), © IAEMEV. CONCLUSION Our authentication 
scheme is a combination of text passwords and graphical passwords. Againfirst round of graphical authentication is a recognition based technique 
whereas second round ofgraphical authentication is a recall based technique. So we have used the term ‘hybrid’ to denoteour scheme. We do believe that 
our design will prove to be more usable and adequately securefor user authentication than existing text-based password and graphical password methods. 
Anobvious and necessary next step is a user study, ideally both a lab study and a field studyleveraging our real-world deployment. The scheme can be 
useful for highly secure systems.Proposed scheme will provide the following advantages: 1. Users’ current sign-in experience is partially preserved. 2. 
A text password alone which is stolen (e.g., by phishing or any other means) does not compromise an account. 3. Random order of images in round-1 
authentication provides a resistance to the shoulder surfing attacks. 4. Password space is very large. 5. It can be implemented in software alone, 
increasing the potential for large-scale adoption on the Internet. Current graphical password techniques are still immature. The field is new and open 
for futureworks.ACKNOWLEDGMENT First and foremost, we offer our gratitude to the almighty Allah whose blessings have helpedus to complete this work. We 
would like to thank Mr. Sarosh Umar, Reader of Aligarh MuslimUniversity for his valuable advice throughout the work. We are also grateful to Mr. 
MohsinAbbas Rizvi, Web developer in Vinove Pvt. Ltd. and Mr. Mohd Anjum, guest faculty at AMUfor their help during the implementation phase. 
REFERENCES[1] William Stallings and Lawrie Brown. Computer Security: Principle and Practices. Pearson Education, 2008.[2] X. Suo, Y. Zhu, and G. S. 
Owen, "Graphical passwords: A survey," 21st Annual Computer Security Applications Conference (ASCSAC 2005). Tucson, 2005.[3] S. Malempati and S. 
Mogalla, “A Well Known Tool Based Graphical Authentication Technique”, CCSEA 2011, CS & IT 02, pp. 97–104, 2011.[4] D.Weinshall and S. Kirkpatrick, 
"Passwords You’ll Never Forget, but Can’t Recall," in Proceedings of Conference on Human Factors in Computing Systems (CHI). Vienna, Austria: ACM, 
2004, pp. 1399-1402.[5] G. E. Blonder, "Graphical passwords," in Lucent Technologies, Inc., Murray Hill, NJ, U. S. Patent-5559961, Ed. United States, 
1996.[6] Susan Wiedenbeck, Jim Waters, Jean-Camille Birget, Alex Brodskiy, and Nasir Memon. Passpoints: design and longitudinal evaluation of a 
graphical password system. International Journal of Human-Computer Studies, 63:102–127, July 2005.[7] Jermyn, I., Mayer A., Monrose, F., Reiter, M., 
and Rubin., “The design and analysis of graphical passwords” in Proceedings of USENIX Security Symposium, August 1999. 586 International Journal of 
Computer Engineering and Technology (IJCET), ISSN 0976 – 6367(Print),ISSN 0976 – 6375(Online) Volume 3, Issue 2, July- September (2012), © IAEME[8] A. 
F. Syukri, E. Okamoto, and M. Mambo, "A User Identification System Using Signature Written with Mouse," in Third Australasian Conference on Information 
Security and Privacy (ACISP): Springer-Verlag Lecture Notes in Computer Science (1438), 1998, pp. 403-441.[9] R. Dhamija, and A. Perrig. “Deja Vu: A 
User Study Using Images for Authentication”. In 9th USENIX Security Symposium, 2000.[10] T. Valentine, "An evaluation of the Passface personal 
authentication system," Technical Report, Goldsmiths College, University of London 1998.[11] L. Sobrado and J.-C. Birget, "Graphical passwords," The 
Rutgers Scholar, An Electronic Bulletin for Undergraduate Research, vol. 4, 2002.[12] S. Man, D. Hong, and M. Mathews, "A shoulder-surfing resistant 
graphical password scheme," in Proceedings of International conference on security and management. Las Vegas, 2003.[13] S. Chiasson, A. Forget, R. 
Biddle, and P.C. van Oorschot. “Influencing Users Towards Better Passwords: Persuasive Cued Click-Points” - In Proc. of HCI’08, September 2008.[14] J. 
Goldberg, J. Hagman, V. Sazawal, "Doodling Our Way To Better Authentication", CHI 02 extended abstracts on Human Factors in Computer Systems, 2002.[15] 
Sabzevar A.P. and Stavrou A., "Universal Multi-Factor Authentication Using Graphical Passwords" - IEEE International Conference on Signal Image 
Technology and Internet Based Systems, SITIS 08, pp. 625-632.Nesar Ahmad is presently working as a Professor in the Department of Computer 
Engineering,Aligarh Muslim University, India. He has nearly twenty years of teaching experience. NesarAhmad obtained his B.Sc (Engg) degree in 
Electronics & Communication Engineering fromBihar College of Engineering, Patna (Now NIT, Patna) in 1984 and M.Sc (InformationEngineering) degree from 
City University, London, U.K., in 1989. He received his Ph.D degreefrom Indian Institute of Technology (IIT), Delhi, India, in 1993. Earlier he worked 
as a SeniorScientific Officer in Microprocessor Applications Program at IIT Delhi. He was with King SaudUniversity, Riyadh during 1997-99 as an 
Assistant Professor. Before joining AMU, he worked asan Assistant Professor in the Department of Electrical Engineering, IIT Delhi till December2004. 
His current research interests mainly include Soft Computing, Web Intelligence and E-Learning. He has authored many papers in various International 
Journals and Conferenceproceedings.Md. Asraful Haque was born in 1985 in West Bengal, India. He received his Master degree inComputer Science and 
Engineering (Specialization-Software Engineering) from Aligarh MuslimUniversity. Presently he is an Assistant Professor (Ad-hoc basis) in Aligarh 
Muslim University.He has more than three years of teaching experience. His area of interests includes Softwareengineering, Operating Systems, Data 
Structure, Image Processing and Internet Security. He hasauthored several papers in different international journals and conferences.Babbar Imam 
received his B.Tech degree in Computer Science and Engineering from AligarhMuslim University in 2009. He is presently pursuing his Master degree in 
Software Engineeringfrom the same University. He has two years of teaching experience. His area of interests includesSoftware engineering, Artificial 
Intelligence, Image Processing and Computer Security.

\section{Non-plaintext Authentication Mechanisms}


See Authentication/Mechanisms for explanation of auth mechanisms. Most installations use only plaintext mechanisms, so you can skip this section unless 
you know you want to use them. The problem with non-plaintext auth mechanisms is that the password must be stored either in plaintext, or using a 
mechanism-specific scheme that's incompatible with all other non-plaintext mechanisms. In addition, the mechanism-specific schemes often offer very 
little protection. This isn't a limitation of Dovecot, it's a requirement for the algorithms to even work. For example if you're going to use CRAM-MD5 
authentication, the password needs to be stored in either PLAIN or CRAM-MD5 scheme. If you want to allow both CRAM-MD5 and DIGEST-MD5, the password 
must be stored in plaintext.The base64 vs. hex encoding that is mentioned above is simply the default encoding that is used. You can override it for 
any scheme by adding a ".hex", ".b64" or ".base64" suffix.This can be especially useful with plaintext passwords to encode characters that would 
otherwise be illegal. For example in passwd-file you couldn't use a ":" character in the password without encoding it to base64 or hex.

\section{URI scheme}

In the field of computer networking, a URI scheme is the top level of the uniform resource identifier (URI) naming structure. All URIs and absolute URI 
references are formed with a scheme name, followed by a colon character (":"), and the remainder of the URI called (in the outdated RFCs 1738 and 2396, 
but not the current STD 66/RFC 3986) the scheme-specific part. The syntax and semantics of the scheme-specific part are left largely to the 
specifications governing individual schemes, subject to certain constraints such as reserved characters and how to "escape" them. URI schemes are 
frequently and incorrectly referred to as "protocols", or specifically as URI protocols or URL protocols, since most were originally designed to be 
used with a particular protocol, and often have the same name. The http scheme, for instance, is generally used for interacting with web resources 
using HyperText Transfer Protocol. Today, URIs with that scheme are also used for other purposes, such as RDF resource identifiers and XML namespaces, 
that are not related to the protocol. Furthermore, some URI schemes are not associated with any specific protocol (e.g. "file") and many others do not 
use the name of a protocol as their prefix. The scheme name consists of a sequence of characters beginning with a letter and followed by any 
combination of letters, digits, plus ("+"), period ("."), or hyphen ("-"). Although schemes are case-insensitive, the canonical form is lowercase and 
documents that specify schemes must do so with lowercase letters. It is followed by a colon.


\stoptext
