\starttext 
The password authentication problem in text based password authentication scheme arises largely from limitations of humans’ long-term memory 
(LTM). ). Once a password has been chosen and learned the user must be able to recall it to log in. If a password is not used frequently it will be 
even more susceptible to forgetting. Graphical password schemes have been proposed as a possible alternative to text-based schemes, motivated 
particularly by the fact that humans can remember pictures better than text. Initially, Blonder (Greg, 1996) give the concept of Graphical User 
Authentication (GUA), one image would appear on the screen whereupon the user would click on a few chosen regions of the image. If the user clicked in 
the correct regions then the user would be authenticated. Most of articles from 1994 till 2009 describe that Graphical Authentication Techniques are 
categorized into three groups:

\section{Pure Recall Based Techniques}

Users reproduce their passwords, without having the chance to use the reminder marks of system. Although easy and convenient, it appears that users do 
not quite remember their passwords. Christopher Varenhorst (1991)[9] proposed a graphical authentication algorithm called passdoodle algorithm. A 
Passdoddle Graphical Authentication algorithm which used the idea of hand written designs or words, drawn with a pen onto a sensitive touchable screen 
was proposed in 2004 by Goldberg and his college. They confirmed that users were able to remember complete doodle images as they would with textual 
passwords.(Christopher, 2004) Weaknesses: According to (Christopher, 2004), people could recall doodle images as accurately as they would at 
alphanumeric passwords. However, such people would not be able to recall the order in which they drew a doodle than the resulting image. On the other 
hand, users were found to be interest by the doodles drawn by other users, and often entered other users’ login details simply to discover a variance 
of the set of doodles from their own (Karen, 2008).

Jermyn Ian et al.(1991)[10] proposed draw secret algorithm. This method consisted of an interface that had a rectangular grid of size G *G, which 
allowed the user to draw a simple picture on a 2D grid. Each cell in this grid is earmarked by discrete rectangular coordinates (x,y). The stroke 
should be a sequence of cells which does not contain a pen up event. Hence the password is defined as some strokes, separated by the pen up event. At 
the time of authentication, the user needs to re-draw the picture by creating the stroke in exactly the same order as in the registration phase. If the 
drawing hits the exact grids and in the same order, the user is authenticated. Weaknesses: Goldberg in his 2002 survey concluded that the majority of 
users could not remember their stroke order. Conversely, the user can recall text passwords faster than they would with DAS Passwords. Yet another 
weakness is that users tend to select extremely weak Graphical Authentications which are susceptible to graphical dictionary attack (Dunphy and Yan, 
2007). Grid selection algorithm increases the DAS password space proposed by Juaie Thorpe, P.C. Van Oorschot.(2004). This research was conducted on the 
complexity of the DAS technique based on password length and stroke count by Thorpe and Orschot. Their study showed that the item which has the 
greatest effect on the DAS password space is the number of strokes. This means that for a fixed password length, if a few strokes are selected then the 
password space will significantly decrease. To enhance security, Thorpe and Orschot created a “Grid Selection” technique. The selection grid has a 
large rectangular region to zoom in on, from the grid which the user selects their key for their password. This definitely increases the DAS password 
space (Muhammad Daniel et al. 2008). Weaknesses: Whilst this method significantly increases the DAS password space, the deficiencies in DAS have not 
been resolved (Muhammad Daniel et al. 2008).

The QDAS method was created by Di Lin, et al.(2007) [11] as a boost to the DAS method, by encoding each stroke. The raw encoding consists of its 
starting cell and the order of qualitative direction change in the stroke vis-a-vie the grid. A directional change is when the pen passes over a cell 
boundary to the direction of the pass in the previous cell boundary. Research has shown that the image which has a hot spot is pivotal as a background 
image (Di et al. 2007). Albeit this model applies dynamic grid transformation to mask the process of creating the password, this method could be safer 
than the original DAS in the fight against shoulder surfing attack and further it has greater entropy than the previous DAS.

In 1998, Syukri et al. proposed a system where authentication is kicked in when the users draw their signatures utilizing the mouse. This technique has 
a two step process, registration and verification. During the registration stage, the user will be required to draw his signature with the mouse, 
whereupon the system will extract the signature area and either enlarge or scale-down the signatures, rotating the same if necessary (Alternatively 
known as normalizing). The information will later be stored in the database. The verification stage initially receives the user input, where upon the 
normalization takes place, and then extracts the parameters of the signature. By using a dynamic updateable database and the geometric average means, 
verification will be performed (Ali Mohamed, 2008). Weaknesses: As not everybody is comfortable with using the mouse as a writing device, the signature 
is so hard to draw, possibly the use of a pen-like input device would resolve this problem. However such devices are not widely used and the addition 
of new hardware to the current system can be expensive (Ali Mohamed, 2008). In this study, researchers concluded that such a technique is more 
pertinent to small devices. 

\section{Cued recall-Based Techniques}

Users reproduce their passwords, without having the chance to use the reminder marks of system. Although easy and convenient, it appears that users do 
not quite remember their passwords.

Greg E. Blonder, in 1966 created a method wherein a pre-determined image is presented to the user on a visual display so that the user should be able 
point to one or more predetermined positions on the image (tap regions) in a predetermined order as a way of pointing out his or her authorization to 
access the resource. Blonder maintained that the method was secure according to the millions of different regions[12]. Weaknesses: The number of 
predefined click regions was relatively small in this algorithm as such the password had to be long for it to be secure. Furthermore, the use of the 
Blonder algorithm necessitates that some special shape similar to a cartoon or artificial image is used in contrast to real pictures (Susan et al. 
2005).

Susan Wiedenbeck, et al.(2005)[13] created the PassPoint algorithm in order to cover the image limitations of the Blonder Algorithm. The picture could 
be any natural picture or painting but at the same time had to be rich enough in order for it to have many possible click points. On the other hand the 
existence of the image has no role other than helping the user to remember the click point. This algorithm has another flexibility which makes it 
possible for there to be no need for artificial pictures which have pre selected regions to be clicked like The Blonder algorithm. During the 
registration phase the user chooses several points on the picture in a certain sequence. To log in, the user only needs to click close to the chosen 
click points, and inside some adjustable tolerable distance, say within 0.25 cm from the actual click point (Susan et al. 2005). The Passpoint system 
has enough features for creating a high entropy algorithm. Since any pixel in the image is a candidate for a click point thus there are hundreds of 
possible memorable points in the challenge image (Ahmet et al. 2007). Weaknesses: The login time, in this method, is longer than in the alphanumeric 
method. Also the user has more difficulty in learning and memorizing in their password. So, users have to go to several trial session for completing 
the process (Susan et al. 2005).

Paul Duaphi(2007)[15,16] created Background DAS Algorithm (BDAS), this method added a background image to the original DAS, such that both the 
background image and the drawing grid is the key to cued recall (Paul et al. 2007). The user begins by trying to have a secret in mind which is made up 
of three points from different categories. Firstly the user starts to draw using the point from a background image. Then the next point of user is that 
the user’s choice of the secret is affected by various characteristics of the image. The last alternative for the user is a mix of the two previous 
methods. Weaknesses: Research on BDAS showed that memory decaying over a week is one of the major obstacles in this algorithm. Users had no issue in 
recreating it in the five-minute test. However, a week later they could not produce the secret password as well as they had done the previous week. 
Further, shoulder-surfing and interference between multiple passwords are concerns for BDAS (Paul et al. 2007).
Passmap algorithm (Roman V. Vamponski,2006) [14] shows a sample of a PassMap password for a passenger who wants to take a trip to Europe as follows: 
One day a tour in Paris around the Eiffel then a tour in London around Big Ben. After these two tours, the third tour will be in Moscow. The passenger 
must be able to visit all of them in a map. Referring to the Figure below, it will be easy to memorize the trip in a map (Roman, 2007).	 Weaknesses: 
The PassMap technology is not very susceptible to "shoulder surfing", but it is susceptible to Brute Force attacks while those mechanisms are great in 
terms of how memorable they are (Roman,2007).

Passlogix Incorporation(Passlogic Inc. Co.,2002) is a commercial security company located in New York City USA. Their scheme, Passlogix v-Go, utilizes 
a technique known as “Repeating a sequence of actions” meaning creating a password in a chronological sequence. Users select their background images 
based on the environment, for example in the kitchen, bathroom, bedroom or others. User can click on a series of items in the image as password. For 
example in the kitchen environment a user can: prepare a meal by selecting a fast food from the refrigerator and put on the hot plate, select some 
vegetables and wash them, then put them on the launch desk (Muhammad Daniel et al. 2008). In case another environment such as the cocktail lounge is 
used, this will allow users to select their favorite vodka, brandy or whiskey and mix it with other cocktails. This type of authentication is easy to 
remember and fun to use (Muhammad Daniel et al. 2008). Weaknesses: Inherently there exist disadvantages such as the size of password space being small. 
After all, the places that one can take vegetables or food from and put into are limited, resulting in the passwords being guessable or predictable 
(Muhammad Daniel et al. 2008).

VisKey(SFR Company,2003) is a one of the recall based authentication schemes commercialized by SFR Company in Germany which was created specifically 
for mobile devices such as PDAs. To form a password, all users need to do is to tap their spots in sequence. Weaknesses: Input tolerance is the major 
drawback of this method. This algorithm permits all input within a certain tolerance area around it, since it is difficult to point to the exact spots 
on the picture. The size of this area can be pre-defined by users. A certain degree of precaution, related to the input precision, needs to be 
exercised, as there is a straight forward correlation between the security and the usability of the password. Practically, the setting of parameters 
with a four spot VisKey theoretically offers almost 1 billion possibilities to define a password. However, such is not large enough to avoid the 
off-line attacks by a high-speed computer. A minimum of seven defined spots are needed in order to overcome the brute force attacks (Muhammad Daniel et 
al. 2008).

In 2006, this scheme being created as an improvement of the DAS algorithm, keeping the advantages of the DAS whilst adding some extra security 
features. Pass- Go is a grid-based scheme which requires a user to select intersections, instead of cells, thus the new system refers to a matrix of 
intersections, rather than cells. Pass-Go Algorithm Changing the format of typing from cells to intersections grants the user more free choices. The 
other difference between these two algorithms is that the size of the grid in the enhanced method changes to 9*9. Weaknesses: The intersections in this 
algorithm do not have boundaries around them, because of this users face error tolerance mechanism. Therefore sensitive areas need to be defined to 
address this problem.

\section{Recognition-Based Techniques}



The fundamental of these techniques is choosing images, icons or symbols from a large collection by the users. During the registration phase, the user 
has to recognize and identify his password image among the bank of decoy images. Research shows that around after one or two months (Saranga and 
Dugald, 2008).

Passface Algorithm(Sacha Brostoff , M. Angela Sasse, 2000)[17] was developed by the idea to choose a face of humans as a password. Firstly, a trial 
session starts with the user in order to have an adventure for the real login process. During the registration phase the user chooses whether their 
image password should be a male or female picture, then chooses four faces from decoy images as the future password. During the login phase, a grid 
which contains nine pictures, is shown to the user. Only one of the user’s passwords among four is shown to user in this grid, and the other eight 
pictures are decoys which are selected from the bank of pictures. Because the password of user contains four faces so the grid repeats continually for 
four times and each repetition contains one of the password pictures. If one of the passwords has been shown in one grid, it will not be shown in the 
next grid. On the other hand the password faces are randomly placed in grids which help to create a more secure environment for the user against 
shoulder-surfing and packet sniffing attacks (Sacha and Angela, 2008). The user tries to identify his four passwords among the other pictures twice in 
a row. According to research, (Ali Mohamed, 2008) this is one of the algorithms which covers most of the usability features like ease of use, and 
straightforward creation and recognition.  Weaknesses: This algorithm like the others suffers from some weaknesses. Firstly, when the password is 
selected by the mouse, it is simple for the attacker to observe the password. The other drawback of this algorithm is the long login time and long 
process through registration phase which causes this algorithm to be slower than textual password authentication
 (Furkan, 2006).

Deja vu Algorithm (Rachna Dhamija, Adrian Perrig, 2000)[18,19] starts by allowing users to select a specific number of pictures from a large image 
portfolio. The pictures are created by random art which is one of hash visualization algorithms. One initial seed is given for starters and then one 
random mathematical formula is generated defining the color value for each pixel in the image. The output will be one random abstract image. The 
benefit of this method is that as the image depends completely on its initial seed, so there is no need for saving the picture pixel by pixel and only 
the seeds need to be stored in the trust server. During authentication phase, the user should pass through a challenging set where his portfolio mixes 
with some decoy images; the user will be authenticated if he is able to identify his password among the entire portfolio. This method causes the 
algorithm to be less vulnerable to description attack. Conversely, the number of pictures in the portfolio and the number of random images could very 
well alter the security of system. Although research has shown that rate of those who use textual passwords, there are several drawbacks with this 
method. Weaknesses: Research on Déjà vu algorithms has shown that it has three soft spots. Firstly, creating a textual password requires 25 seconds but 
with this method a user needs about 60 seconds to create the password. Secondly, the process of selecting pictures from the database can be tedious and 
time consuming for the user. Finally, the password seeds for each user can just be saved in the plain text format (Rachna, 2000). Leonardo Sobrado , 
J-Canille Birget ( 2002 ) [20] proposed the triangle algorithm based on several schemes to resist the Shoulder surfing attack. The first scheme named 
triangle, randomly places a set of N objects (a few hundred or a few thousand) on the screen. Additionally, there is a subset of K pass objects 
previously chosen and memorized by the user. The system will select the placement of N objects randomly in the log-in phase The system initially 
chooses a patch randomly covering half the screen, and then randomly again places the K password objects in that patch. In the log-in phase, the user 
must be able to find the location of three pass-objects and then click inside the invisible triangle that is possible to create those three objects. 
But, for each login this process will be repeated using a different group of n objects. So, it is possible to say that there is a very low probability 
of randomly clicking in the correct area (Leonardo et al. 2002). Weaknesses: The log-in phase must use a minimum of 1000 images in order to resist 
shoulder surfing. As a result the log-in will be very crowded and the password will be indistinguishable (Xiaoyuan, 2005).

The moveable frame algorithm proposed by Leonardo Sobrado , J-Canille Birget in 2002[20] had a similar idea to that of triangle method. However in its 
case the user had to select three objects from K objects in the login phase. Only 3 pass objects are displayed at any given time and only one of them 
is placed in a movable frame. The user must move the frame until the three objects line up one after the other. These operations minimize the random 
movements involved in finding the password (Leonardo et al. 2002). Weaknesses: Just like in the triangle algorithm, there are many objects involved in 
this algorithm which can lead to the user being unsatisfied and in most cases will confuse users (Leonardo et al. 2002).

Picture Password Algorithm (Wayne Jansen et al, 2003) [21] was designed especially for handheld devices like Personal Digital Assistant (PDA). During 
enrollment, the user selecting a theme identifying the thumbnail photos to be applied and then registers a sequence of thumbnail images that are used 
as a future password. If the device is powered on, then the user must input the true sequence of images but after successful log-in the user can change 
the password. In this algorithm the password space will be small because the number of photos is limited to 30. In order to solve this problem, the 
designer added a second step to the algorithm. This means the user can select two thumbnails together to compose the new alphabet element by using a 
shift key to select uppercase or special characters. Weaknesses: The memorability will be more complex when the second part which solves the password 
space’s problem is added to the algorithm (Wayne et al. 2003).

The Story Algorithm (Darren Davies, et al. 2004) [23] that was proposed to categorized the available pictures into nine categories namely animals, 
cars, women, foods, children, men, objects, natures and sports. This algorithm was proposed by Carnegie Mellon University to be used for different 
purposes. In this method the user selects the password from the mixed pictures in the nine categories in order to make a story (Darren et al. 2004). 
Weaknesses: Research showed that the story scheme was difficult to commit to memory in comparison to pass face authentication.

The moveable frame algorithm proposed by Leonardo Sobrado , J-Canille Birget in 2002[20] had a similar idea to that of triangle method. However in its 
case the user had to select three objects from K objects in the login phase. Only 3 pass objects are displayed at any given time and only one of them 
is placed in a movable frame. The user must move the frame until the three objects line up one after the other. These operations minimize the random 
movements involved in finding the password (Leonardo et al. 2002). Weaknesses: Just like in the triangle algorithm, there are many objects involved in 
this algorithm which can lead to the user being unsatisfied and in most cases will confuse users (Leonardo et al. 2002).

Picture Password Algorithm (Wayne Jansen et al, 2003) [21] was designed especially for handheld devices like Personal Digital Assistant (PDA). During 
enrollment, the user selecting a theme identifying the thumbnail photos to be applied and then registers a sequence of thumbnail images that are used 
as a future password. If the device is powered on, then the user must input the true sequence of images but after successful log-in the user can change 
the password. In this algorithm the password space will be small because the number of photos is limited to 30. In order to solve this problem, the 
designer added a second step to the algorithm. This means the user can select two thumbnails together to compose the new alphabet element by using a 
shift key to select uppercase or special characters. Weaknesses: The memorability will be more complex when the second part which solves the password 
space’s problem is added to the algorithm (Wayne et al. 2003).

The Story Algorithm (Darren Davies, et al. 2004) [23] that was proposed to categorized the available pictures into nine categories namely animals, 
cars, women, foods, children, men, objects, natures and sports. This algorithm was proposed by Carnegie Mellon University to be used for different 
purposes. In this method the user selects the password from the mixed pictures in the nine categories in order to make a story (Darren et al. 2004). 
Weaknesses: Research showed that the story scheme was difficult to commit to memory in comparison to pass face authentication.


In order to offer resistance against shoulder surfing, in 2003 another WIW algorithm was proposed (Shushuang Man, et al. 2003) [22], which uses a 
unique code for each picture. The user selects some picture as a password. This picture must be found in the log-in phase before the user can type the 
related unique code in a text box. The argument is that it is very hard to dismantle this kind of password even if the whole authentication process is 
recorded on video as there is no mouse click to give away the pass-object information. Weaknesses: One of the main vulnerabilities of this algorithm is 
memorizing the alphanumeric code for each password by the user.

\section{GRAPHICAL ALTERNATIVES FOR USER AUTHENTICATION}

 Authenticating users by means of passwords is still the dominant form of authentication 
despite its recognised weaknesses. To solve this, authenticating users with images or pictures (i.e. graphical passwords) is proposed as one possible 
alternative as it is claimed that pictures are easy to remember, easy to use and has considerable security. Reviewing literature from the last twenty 
years found that few graphical password schemes have successfully been applied as the primary user authentication mechanism, with many studies 
reporting that their proposed scheme was better than their predecessors and they normally compared their scheme with the traditional password-based. In 
addition, opportunities for further research in areas such as image selection, image storage and retrieval, memorability (i.e. the user’s ability to 
remember passwords), predictability, applicability to multiple platforms, as well as users’ familiarity are still widely possible. Motivated by the 
above findings and hoping to reduce the aforementioned issues, this thesis reports upon a series of graphical password studies by comparing existing 
methods, developing a novel alternative scheme, and introducing guidance for users before they start selecting their password. Specifically, two 
studies comparing graphical password methods were conducted with the specific aims to evaluate users’ familiarity and perception towards graphical 
methods and to examine the performance of graphical methods in the web environment. To investigate the feasibility of combining two graphical methods, 
a novel graphical method known as EGAS (Enhanced Graphical Authentication System) was developed and tested in terms of its ease of use, ideal secret 
combination, ideal login strategies, effect of using smaller tolerances (i.e. areas where the click is still accepted) as well as users’ familiarity. 
In addition, graphical password guidelines (GPG) were introduced and deployed within the EGAS prototype, in order to evaluate their potential to assist 
users in creating appropriate password choices.
 From these studies, the thesis provides an alternative classification for graphical password methods by looking at the users’ tasks when 
authenticating into the system; namely click-based, choice-based, draw-based and hybrid. Findings from comparative studies revealed that although a 
number of participants stated that they were aware of the existence of graphical passwords, they actually had little understanding of the methods 
involved. Moreover, the methods of selecting a series of images (i.e. choice-based) and clicking on the image (i.e. click-based) are actually possible 
to be used for web-based authentication due to both of them reporting complementary results. With respect to EGAS, the studies have shown that 
combining two graphical methods is possible and does not introduce negative effects upon the resulting usability. User familiarity with the EGAS 
software prototype was also improved as they used the software for periods of time, with improvement shown in login time, accuracy and login failures.
 With the above findings, the research proposes that users’ familiarity is one of the key elements in deploying any graphical method, and appropriate 
HCI guidelines should be considered and employed during development of the scheme. Additionally, employing the guidelines within the graphical method 
and not treating them as a separate entity in user authentication is also recommended. Other than that, elements such as reducing predictability, 
testing with multiple usage scenarios and platforms, as well as flexibility with respect to tolerance should be the focus for future research.


\section{Token-based Graphical Password Authentication}



Given that phishing is an ever increasing problem, a better authentication system than the current alphanumeric system is needed. Because of the large 
number of current authentication systems that use alphanumeric passwords, a new solution should be compatible with these systems. We propose a system 
that uses a graphical password deployed from a Trojan and virus resistant embedded device as a possible solution. The graphical password would require 
the user to choose a family photo sized to 441x331 pixels. Using this image, a novel, image hash provides an input into a cryptosystem on the embedded 
device that subsequently returns an encryption key or text password. The graphical password requires the user to click five to eight points on the 
image. From these click-points, the embedded device stretches the graphical password input to a 32- character, random, unique alphanumeric password or 
a 256-bit AES key. Each embedded device and image are unique components in the graphical password system. Additionally, one graphical password can 
generate many 32-character unique, alphanumeric passwords using its embedded device which eliminates the need for the user to memorize many passwords.



\stoptext
